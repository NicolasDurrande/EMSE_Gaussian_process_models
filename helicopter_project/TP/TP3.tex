\documentclass[a4paper,10pt]{article}

\usepackage[utf8]{inputenc}
\usepackage[T1]{fontenc}
\usepackage[francais]{babel}
\usepackage{amsmath}
% \usepackage{verbatim}
\usepackage{fancyvrb}
\usepackage[top=3cm, bottom=3cm, left=3cm , right=3cm]{geometry}

\renewcommand{\baselinestretch}{1.1}

\title{\vspace{-1cm} TP3 Projet hélicoptère : Optimisation}
\author{Mines Saint-\'Etienne, majeure Data Science,  2016\:-\:2017 }
\date{}

\begin{document}
\maketitle
\paragraph{Objectif :} effectuer quelques itérations d'EGO afin de trouver l'hélicoptère qui maximise le temps de chute.

\subsection*{Amélioration espérée dans le cas bruité}
L'algorithme EGO vous semble-t-il adapté au cas d'observations bruitées ? Consultez la documentation du package \texttt{DiceOptim} pour trouver des solutions à ce problème. 

\subsection*{Optimisation}
Trouvez la configuration des paramètres d'entrée qui maximise l'équivalent de l'EI dans le cas bruité, effectuez l'expérience associée. Le temps de chute se trouve-t-il amélioré? Recommencez plusieurs fois la procédure, en mettant à jour votre modèle de krigeage à chaque fois. 

\subsection*{Rapport}
\paragraph{}
Finalisez l'écriture de votre rapport, en prenant soin de justifier l'ensemble des choix que vous avez été amenés à faire au cours du projet. N'oubliez pas d'indiquer dans votre rapport la configuration optimale que vous avez trouvée.

\end{document}