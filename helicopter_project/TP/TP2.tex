\documentclass[a4paper,10pt]{article}

\usepackage[utf8]{inputenc}
\usepackage[T1]{fontenc}
\usepackage[francais]{babel}
\usepackage{amsmath}
\usepackage[top=3cm, bottom=3cm, left=3cm , right=3cm]{geometry}

\renewcommand{\baselinestretch}{1.1}

\title{\vspace{-1cm} TP2 Projet hélicoptère : construction de modèles de krigeage}
\author{Mines Saint-\'Etienne, majeure Data Science,  2016\:-\:2017 }
\date{}

\begin{document}
\maketitle
Le but de ce TP est de construire un modèle statistique du temps de chute à partir des expériences qui ont été réalisées. Le meilleur modèle obtenu sera utilisé pour les séances suivantes.

\subsection*{Réflexions préliminaires}
Avant de vous lancer dans la création de modèles, effectuez une première analyse des données. De plus, prenez le temps de vous poser des questions telles que :
\begin{itemize}
	\item Un pré-traitement des données est-il nécessaire (données aberrantes, transformation log, ...) ?
 	\item Est-il pertinent de changer la paramétrisation du problème ? 
 	\item ...
\end{itemize} 

\subsection*{Construction de modèles de krigeage (1h30)}

Le package \texttt{DiceKriging} vous permettra de gagner un temps précieux. La construction de modèles se fait alors à l'aide de la fonction \texttt{km}, et les prédictions à l'aide de la fonction  \texttt{predict}. De plus, ce package possède déjà des outils de diagnostic des modèles (\texttt{plot(model), summary(model)}) et il permet d'estimer les paramètres des noyaux par maximum de vraisemblance ou par validation croisée. Par ailleurs, le package \texttt{DiceView} peut être utile pour obtenir des représentations graphiques de modèles.

\paragraph{}
Comparez différents modèles en essayant différents noyaux et différentes tendances. Ces modèles permettent-ils d'obtenir de meilleurs résultats qu'un simple modèle de régression linéaire ?

\paragraph{}
Choisissez le modèle qui vous semble le mieux adapté et donnez des arguments vous ayant amenés à faire ce choix. Attention, cela a-t-il du sens d'appliquer des méthodes \emph{leave-one-out} pour juger de la qualité d'un modèle ?
Si le temps le permet, analysez la robustesse des estimations des paramètres des noyaux.

\paragraph{}
Interprétez les valeurs des paramètres du modèle retenu (bruit d'observation, portées, ...). Si vous choisissez d'utiliser des termes de tendance, peuvent-ils s'interpréter ?

\subsection*{Région contenant l'optimum (1h30)}
\paragraph{}
À partir du modèle dont vous disposez, êtes vous capable d'identifier une région contenant \emph{a priori} l'optimum ? Quelle méthode pouvez-vous proposer pour enrichir le plan d'expériences dans cette région ? Réalisez 10 nouvelles expériences pour affiner le modèle dans cette région d'intérêt.
\end{document}



















\section*{Aide : fonctions et packages \texttt{R} utiles}

\begin{itemize}
	\item construire un modèle de krigeage : package \texttt{DiceKriging}, fonctions \texttt{km, predict.km})	
	\item pour l'analyse des modèles : \texttt{plot(model), print(model), summary(model)}
\end{itemize}

\end{document}


\begin{Verbatim}[frame=single,fontsize==\tiny]
generateCVT <- function(npts, dim, nite)
\end{Verbatim}
Le plan d'expériences s'écrira comme une matrice de $npts$ lignes et $dim$ colonnes.

\subsection{Hypercubes latins}
Créez un script R permettant de générer un hypercube latin aléatoire (normalisé) pour un nombre quelconque de points et une dimension quelconque.

\begin{Verbatim}[frame=single,fontsize==\tiny]
generateLHS <- function(npts, dim)
\end{Verbatim}

\subsection{Critères}
Créez un script R permettant de calculer le critère maximin d'un plan. Le script fournira également en sortie la matrice des interdistances (taille $npts \times npts$). Attention à éviter les calculs redondants (la matrice est symétrique !).

\begin{Verbatim}[frame=single,fontsize==\tiny]
evalMinDist <- function(X)
{ (.....)
return( list(minDist = d, allDist = D) ) }
\end{Verbatim}

\subsection{Hypercubes latins optimisés}
Utilisez le script précédent pour construire des hypercubes latins optimisés. Selon votre vitesse d'avancement, vous coderez au choix :
\begin{itemize}
 \item une recherche aléatoire pure : on génère un certain nombre de plans et on garde le meilleur ;
 \item un algorithme d'échange simple : on génère un plan, puis on propose des échanges aléatoires qu'on accepte s'ils améliorent le critère ;
 \item un algorithme d'échange avec recuit simulé.
\end{itemize}
Dans les 2 derniers cas, les performances peuvent être améliorées en choisissant pour l'échange un point critique pour le critère.

\section{Analyse}
Générez et comparez des plans CVT, LHS quelconque et LHS optimisé pour les configurations suivantes :
\begin{itemize}
 \item dimension 2, 10 points ; 
 \item dimension 4, 40 points ;
 \item dimension 10, 150 points.
\end{itemize}

Observez en particulier les différences sur :
\begin{itemize}
 \item la répartition sur les marginales de dimension 1 (histogrammes)
 \item la répartition sur les marginales de dimension 2 (fonction \texttt{pairs})
 \item la valeur du critère \textit{maximin}, éventuellement l'histogramme des interdistances.
\end{itemize}
Peut-on déterminer facilement les ``meilleurs'' plans ? Essayez éventuellement d'autres critères / visualisations afin de faciliter votre choix.

\section{Préparation du projet : plan de taille $4 \times 40$}
Vous aurez besoin, pour le projet, d'un plan d'expériences de 40 points en dimension 4. Sauvegardez le ``meilleur'' plan que vous ayez généré selon la méthode de votre choix.
\textbf{Ce plan est nécessaire pour la suite des TPs}.

NB : Les éléments vous ayant conduit à choisir votre plan doivent figurer dans le rapport.

\subsection*{Aide : fonctions et packages \texttt{R} utiles}

\begin{itemize}
	\item pour générer des permutations : \texttt{sample.int}
	\item pour les modèles linéaires : \texttt{lm}
	\item pour le krigeage : package \texttt{DiceKriging} (fonctions \texttt{km, predict.km})	
	\item pour la représentation 3D : \texttt{scatterplot3d} (dans \texttt{library("scatterplot3d")}), \texttt{persp}, \texttt{contour}
	\item pour l'analyse des modèles : \texttt{plot(model), print(model), summary(model)}
\end{itemize}

\end{document}